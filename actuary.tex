\documentclass[12pt]{beamer}
\usepackage[T2A]{fontenc}
\usepackage[utf8]{inputenc}
\usepackage[english,russian]{babel}
\usepackage{amssymb,amsfonts,amsmath,mathtext}
\usepackage{cite,enumerate,float}
\usepackage{pgfplots}
\pgfplotsset{compat=newest}
\usepackage{lipsum}

\setlength{\parindent}{3ex}
\setlength{\parskip}{0.5em}

\usetheme{Pittsburgh}
\usecolortheme{whale}

\title[Модель коллективного риска]{Вероятность разорения в классической модели коллективного риска} 
\author{Кодзоев М., Куркин М., Шаповалов Р.}
\institute[ВМК МГУ]
{
Московский Государственный Университет им. Ломоносова
}
\date{\today} % Date, can be changed to a custom date

\begin{document}

\begin{frame}
\titlepage
\end{frame}

\begin{frame}
\tableofcontents
\end{frame}

\begin{frame}
\section{Описание модели}
\frametitle{Модель с непрерывным временем} 
%число страховых случаев и S(t) - суммарные страховые выплаты, произведенные до момента t. Начинаем отсчет с нуля.
Используем два случайных процесса с непрерывным временем: процесс числа страховых случаев  $\big\{N(t), t\geq 0\big\}$  и процесс суммарных выплат $\big\{S(t), t\geq 0\big\}$ 

Пусть $\exists \: N(t)\: $ - число страховых случаев и $S(t)\:  $ - суммарные страховые выплаты, произведенные до момента t. Начинаем отсчет с нуля.

Пусть$ X_{i} \:-$ величина $i$-ой страховой выплаты. Тогда $S(t) = X_{1}+X_{2}+...+X_{N(t)}$

Пусть $t\geq 0$ и $h> 0$ . Тогда разность $N(t+h)-N(t)$ является числом страховых случаев, а разность $S(t+h) -S(t)$ - суммарными страховыми выплатами, которые происходят в интервале между $t$ и $t + h$.


\end{frame}


\begin{frame}
\frametitle{Модель с непрерывным временем} 
Пусть $T_{i}$ обозначает момент времени, когда происходит $i$-ый страховой случай. Тогда $T_{1}, T_{2}, ...$ - случайные величины. Чтобы исключить возможность возникновения одновременно двух и более страховых случаев, будем считать, что $T_{1}<T_{2}<T_{3}< . . . $

Пусть $V_{1} = T_{1}$ и $V_{i} = T_{i}-T_{i-1}, i>1$ время, прошедшее между двумя последовательными страховыми случаями  






\tikzset{every picture/.style={line width=0.75pt}} %set default line width to 0.75pt        

\begin{tikzpicture}[x=0.75pt,y=0.75pt,yscale=-1,xscale=1]
%uncomment if require: \path (0,242.25); %set diagram left start at 0, and has height of 242.25

%Straight Lines [id:da6902289836929021] 
\draw [color={rgb, 255:red, 0; green, 0; blue, 0 }  ,draw opacity=1 ]   (59.99,111.53) -- (83.17,111.69) ;

\draw [shift={(59.99,111.53)}, rotate = 0.39] [color={rgb, 255:red, 0; green, 0; blue, 0 }  ,draw opacity=1 ][fill={rgb, 255:red, 0; green, 0; blue, 0 }  ,fill opacity=1 ][line width=0.75]      (0, 0) circle [x radius= 3.35, y radius= 3.35]   ;
%Straight Lines [id:da9313950171592851] 
\draw [color={rgb, 255:red, 0; green, 0; blue, 0 }  ,draw opacity=1 ][line width=1.5]    (17.26,156.62) -- (39.01,156.62) ;

\draw [shift={(17.26,156.62)}, rotate = 0] [color={rgb, 255:red, 0; green, 0; blue, 0 }  ,draw opacity=1 ][fill={rgb, 255:red, 0; green, 0; blue, 0 }  ,fill opacity=1 ][line width=1.5]      (0, 0) circle [x radius= 4.36, y radius= 4.36]   ;
%Straight Lines [id:da0611422079179329] 
\draw [color={rgb, 255:red, 0; green, 0; blue, 0 }  ,draw opacity=1 ]   (84.75,90.01) -- (106.08,90.17) ;

\draw [shift={(84.75,90.01)}, rotate = 0.43] [color={rgb, 255:red, 0; green, 0; blue, 0 }  ,draw opacity=1 ][fill={rgb, 255:red, 0; green, 0; blue, 0 }  ,fill opacity=1 ][line width=0.75]      (0, 0) circle [x radius= 3.35, y radius= 3.35]   ;
%Shape: Axis 2D [id:dp048941842347560494] 
\draw [color={rgb, 255:red, 155; green, 155; blue, 155 }  ,draw opacity=1 ][line width=1.5]  (7,156.62) -- (139.5,156.62)(17.26,49.25) -- (17.26,166.73) (132.5,151.62) -- (139.5,156.62) -- (132.5,161.62) (12.26,56.25) -- (17.26,49.25) -- (22.26,56.25) (39.26,151.62) -- (39.26,161.62)(61.26,151.62) -- (61.26,161.62)(83.26,151.62) -- (83.26,161.62)(105.26,151.62) -- (105.26,161.62)(12.26,134.62) -- (22.26,134.62)(12.26,112.62) -- (22.26,112.62)(12.26,90.62) -- (22.26,90.62)(12.26,68.62) -- (22.26,68.62) ;
\draw   (46.26,168.62) node[anchor=east, scale=0.75]{$T_{1}$} (68.26,168.62) node[anchor=east, scale=0.75]{$T_{2}$} (90.26,168.62) node[anchor=east, scale=0.75]{$T_{3}$} (112.26,168.62) node[anchor=east, scale=0.75]{$T_{4}$} (14.26,134.62) node[anchor=east, scale=0.75]{1} (14.26,112.62) node[anchor=east, scale=0.75]{2} (14.26,90.62) node[anchor=east, scale=0.75]{3} (14.26,68.62) node[anchor=east, scale=0.75]{4} ;
%Straight Lines [id:da7400584581833622] 
\draw    (39.21,134.57) -- (61.99,134.57) ;

\draw [shift={(39.21,134.57)}, rotate = 0] [color={rgb, 255:red, 0; green, 0; blue, 0 }  ][fill={rgb, 255:red, 0; green, 0; blue, 0 }  ][line width=0.75]      (0, 0) circle [x radius= 3.35, y radius= 3.35]   ;
%Straight Lines [id:da7427758926081622] 
\draw    (109.86,68.5) -- (140.11,68.65) ;

\draw [shift={(109.86,68.5)}, rotate = 0.3] [color={rgb, 255:red, 0; green, 0; blue, 0 }  ][fill={rgb, 255:red, 0; green, 0; blue, 0 }  ][line width=0.75]      (0, 0) circle [x radius= 3.35, y radius= 3.35]   ;
%Straight Lines [id:da4366265791607704] 
\draw [color={rgb, 255:red, 0; green, 0; blue, 0 }  ,draw opacity=1 ]   (296.99,111.05) -- (320.17,111.21) ;

\draw [shift={(296.99,111.05)}, rotate = 0.39] [color={rgb, 255:red, 0; green, 0; blue, 0 }  ,draw opacity=1 ][fill={rgb, 255:red, 0; green, 0; blue, 0 }  ,fill opacity=1 ][line width=0.75]      (0, 0) circle [x radius= 3.35, y radius= 3.35]   ;
%Straight Lines [id:da35568379619841295] 
\draw [color={rgb, 255:red, 0; green, 0; blue, 0 }  ,draw opacity=1 ][line width=1.5]    (254.26,156.14) -- (276.01,156.14) ;

\draw [shift={(254.26,156.14)}, rotate = 0] [color={rgb, 255:red, 0; green, 0; blue, 0 }  ,draw opacity=1 ][fill={rgb, 255:red, 0; green, 0; blue, 0 }  ,fill opacity=1 ][line width=1.5]      (0, 0) circle [x radius= 4.36, y radius= 4.36]   ;
%Straight Lines [id:da489846144654176] 
\draw [color={rgb, 255:red, 0; green, 0; blue, 0 }  ,draw opacity=1 ]   (321.75,89.53) -- (343.08,89.69) ;

\draw [shift={(321.75,89.53)}, rotate = 0.43] [color={rgb, 255:red, 0; green, 0; blue, 0 }  ,draw opacity=1 ][fill={rgb, 255:red, 0; green, 0; blue, 0 }  ,fill opacity=1 ][line width=0.75]      (0, 0) circle [x radius= 3.35, y radius= 3.35]   ;
%Shape: Axis 2D [id:dp8213897888455484] 
\draw [color={rgb, 255:red, 155; green, 155; blue, 155 }  ,draw opacity=1 ][line width=1.5]  (244,156.14) -- (376.5,156.14)(254.26,48.77) -- (254.26,166.25) (369.5,151.14) -- (376.5,156.14) -- (369.5,161.14) (249.26,55.77) -- (254.26,48.77) -- (259.26,55.77) (276.26,151.14) -- (276.26,161.14)(298.26,151.14) -- (298.26,161.14)(320.26,151.14) -- (320.26,161.14)(342.26,151.14) -- (342.26,161.14)(249.26,134.14) -- (259.26,134.14)(249.26,112.14) -- (259.26,112.14)(249.26,90.14) -- (259.26,90.14)(249.26,68.14) -- (259.26,68.14) ;
\draw   (283.26,168.14) node[anchor=east, scale=0.75]{$T_{1}$} (305.26,168.14) node[anchor=east, scale=0.75]{$T_{2}$} (327.26,168.14) node[anchor=east, scale=0.75]{$T_{3}$} (349.26,168.14) node[anchor=east, scale=0.75]{$T_{4}$} (251.26,134.14) node[anchor=east, scale=0.75]{} (251.26,112.14) node[anchor=east, scale=0.75]{} (251.26,90.14) node[anchor=east, scale=0.75]{} (251.26,68.14) node[anchor=east, scale=0.75]{} ;
%Straight Lines [id:da7798685322144794] 
\draw    (276.21,134.09) -- (298.99,134.09) ;

\draw [shift={(276.21,134.09)}, rotate = 0] [color={rgb, 255:red, 0; green, 0; blue, 0 }  ][fill={rgb, 255:red, 0; green, 0; blue, 0 }  ][line width=0.75]      (0, 0) circle [x radius= 3.35, y radius= 3.35]   ;
%Straight Lines [id:da4474268441144118] 
\draw    (346.86,68.02) -- (377.11,68.18) ;

\draw [shift={(346.86,68.02)}, rotate = 0.3] [color={rgb, 255:red, 0; green, 0; blue, 0 }  ][fill={rgb, 255:red, 0; green, 0; blue, 0 }  ][line width=0.75]      (0, 0) circle [x radius= 3.35, y radius= 3.35]   ;
%Straight Lines [id:da33465736176175764] 
\draw    (21,183.02) -- (37.5,183.23) ;
\draw [shift={(39.5,183.25)}, rotate = 180.7] [fill={rgb, 255:red, 0; green, 0; blue, 0 }  ][line width=0.75]  [draw opacity=0] (8.93,-4.29) -- (0,0) -- (8.93,4.29) -- cycle    ;
\draw [shift={(19,183)}, rotate = 0.7] [fill={rgb, 255:red, 0; green, 0; blue, 0 }  ][line width=0.75]  [draw opacity=0] (8.93,-4.29) -- (0,0) -- (8.93,4.29) -- cycle    ;
%Straight Lines [id:da18440378457805173] 
\draw    (41.5,183.27) -- (58,183.48) ;
\draw [shift={(60,183.5)}, rotate = 180.7] [fill={rgb, 255:red, 0; green, 0; blue, 0 }  ][line width=0.75]  [draw opacity=0] (8.93,-4.29) -- (0,0) -- (8.93,4.29) -- cycle    ;
\draw [shift={(39.5,183.25)}, rotate = 0.7] [fill={rgb, 255:red, 0; green, 0; blue, 0 }  ][line width=0.75]  [draw opacity=0] (8.93,-4.29) -- (0,0) -- (8.93,4.29) -- cycle    ;
%Straight Lines [id:da10196844617549994] 
\draw    (62,183.52) -- (81.5,183.73) ;
\draw [shift={(83.5,183.75)}, rotate = 180.61] [fill={rgb, 255:red, 0; green, 0; blue, 0 }  ][line width=0.75]  [draw opacity=0] (8.93,-4.29) -- (0,0) -- (8.93,4.29) -- cycle    ;
\draw [shift={(60,183.5)}, rotate = 0.61] [fill={rgb, 255:red, 0; green, 0; blue, 0 }  ][line width=0.75]  [draw opacity=0] (8.93,-4.29) -- (0,0) -- (8.93,4.29) -- cycle    ;
%Straight Lines [id:da5079566975875298] 
\draw    (85.5,183.77) -- (105,183.98) ;
\draw [shift={(107,184)}, rotate = 180.61] [fill={rgb, 255:red, 0; green, 0; blue, 0 }  ][line width=0.75]  [draw opacity=0] (8.93,-4.29) -- (0,0) -- (8.93,4.29) -- cycle    ;
\draw [shift={(83.5,183.75)}, rotate = 0.61] [fill={rgb, 255:red, 0; green, 0; blue, 0 }  ][line width=0.75]  [draw opacity=0] (8.93,-4.29) -- (0,0) -- (8.93,4.29) -- cycle    ;
%Straight Lines [id:da7528743894355441] 
\draw    (107,184.01) -- (136.5,184.24) ;
\draw [shift={(138.5,184.25)}, rotate = 180.43] [fill={rgb, 255:red, 0; green, 0; blue, 0 }  ][line width=0.75]  [draw opacity=0] (8.93,-4.29) -- (0,0) -- (8.93,4.29) -- cycle    ;
\draw [shift={(105,184)}, rotate = 0.43] [fill={rgb, 255:red, 0; green, 0; blue, 0 }  ][line width=0.75]  [draw opacity=0] (8.93,-4.29) -- (0,0) -- (8.93,4.29) -- cycle    ;
%Straight Lines [id:da7319537544331218] 
\draw    (235.5,135.25) -- (235.5,154.25) ;
\draw [shift={(235.5,156.25)}, rotate = 270] [fill={rgb, 255:red, 0; green, 0; blue, 0 }  ][line width=0.75]  [draw opacity=0] (8.93,-4.29) -- (0,0) -- (8.93,4.29) -- cycle    ;
\draw [shift={(235.5,133.25)}, rotate = 90] [fill={rgb, 255:red, 0; green, 0; blue, 0 }  ][line width=0.75]  [draw opacity=0] (8.93,-4.29) -- (0,0) -- (8.93,4.29) -- cycle    ;
%Straight Lines [id:da20330492195486727] 
\draw    (235.5,112.25) -- (235.5,131.25) ;
\draw [shift={(235.5,133.25)}, rotate = 270] [fill={rgb, 255:red, 0; green, 0; blue, 0 }  ][line width=0.75]  [draw opacity=0] (8.93,-4.29) -- (0,0) -- (8.93,4.29) -- cycle    ;
\draw [shift={(235.5,110.25)}, rotate = 90] [fill={rgb, 255:red, 0; green, 0; blue, 0 }  ][line width=0.75]  [draw opacity=0] (8.93,-4.29) -- (0,0) -- (8.93,4.29) -- cycle    ;
%Straight Lines [id:da38445821520516876] 
\draw    (235.5,89.25) -- (235.5,108.25) ;
\draw [shift={(235.5,110.25)}, rotate = 270] [fill={rgb, 255:red, 0; green, 0; blue, 0 }  ][line width=0.75]  [draw opacity=0] (8.93,-4.29) -- (0,0) -- (8.93,4.29) -- cycle    ;
\draw [shift={(235.5,87.25)}, rotate = 90] [fill={rgb, 255:red, 0; green, 0; blue, 0 }  ][line width=0.75]  [draw opacity=0] (8.93,-4.29) -- (0,0) -- (8.93,4.29) -- cycle    ;
%Straight Lines [id:da9780503776830265] 
\draw    (235.5,66.25) -- (235.5,85.25) ;
\draw [shift={(235.5,87.25)}, rotate = 270] [fill={rgb, 255:red, 0; green, 0; blue, 0 }  ][line width=0.75]  [draw opacity=0] (8.93,-4.29) -- (0,0) -- (8.93,4.29) -- cycle    ;
\draw [shift={(235.5,64.25)}, rotate = 90] [fill={rgb, 255:red, 0; green, 0; blue, 0 }  ][line width=0.75]  [draw opacity=0] (8.93,-4.29) -- (0,0) -- (8.93,4.29) -- cycle    ;
%Straight Lines [id:da8735758648256122] 
\draw    (235.5,43.25) -- (235.5,62.25) ;
\draw [shift={(235.5,64.25)}, rotate = 270] [fill={rgb, 255:red, 0; green, 0; blue, 0 }  ][line width=0.75]  [draw opacity=0] (8.93,-4.29) -- (0,0) -- (8.93,4.29) -- cycle    ;
\draw [shift={(235.5,41.25)}, rotate = 90] [fill={rgb, 255:red, 0; green, 0; blue, 0 }  ][line width=0.75]  [draw opacity=0] (8.93,-4.29) -- (0,0) -- (8.93,4.29) -- cycle    ;

% Text Node
\draw (721,21) node   {$0$};
% Text Node
\draw (701,71) node   {$0$};
% Text Node
\draw (29.62,36.97) node  [align=left] {$N(t)$};
% Text Node
\draw (146.87,167.28) node  [align=left] {$t$};
% Text Node
\draw (266.62,36.49) node  [align=left] {$S(t)$};
% Text Node
\draw (383.87,162.8) node  [align=left] {$t$};
% Text Node
\draw (29.62,197.97) node  [align=left] {$V_{1}$};
% Text Node
\draw (49.62,198.25) node  [align=left] {$V_{2}$};
% Text Node
\draw (70.62,198.25) node  [align=left] {$V_{3}$};
% Text Node
\draw (94.62,198.25) node  [align=left] {$V_{4}$};
% Text Node
\draw (122.62,198.25) node  [align=left] {$V_{5}$};
% Text Node
\draw (217.62,145.25) node  [align=left] {$X_{1}$};
% Text Node
\draw (217.62,122.25) node  [align=left] {$X_{2}$};
% Text Node
\draw (217.62,97.25) node  [align=left] {$X_{3}$};
% Text Node
\draw (217.62,78.25) node  [align=left] {$X_{4}$};
% Text Node
\draw (217.62,55.25) node  [align=left] {$X_{5}$};


\end{tikzpicture}

\end{frame}

\begin{frame}
\frametitle{Модель с непрерывным временем} 
Имеется несколько способов определять распределение процесса числа страховых случаев. 
Рассмотрим два из них:

(1) \textit{Глобальный метод.} Для всех $t\geq 0$ и всех $h>0$ мы определяем условное распределение c.в. $N(t + h) - N(t)$ при условии, что значения с.в. $N(s)$ для $s \leq t$ известны.

(2) \textit{Дискретный метод.} Мы определяем совместное распределение с.в. $V_{1}, V_{2}, V_{3}, ...$

\setlength{\parskip}{2.5em}
Мы переходим к пуассоновскому процессу, для которого длины интервала времени между последовательными страховыми случаями взаимно независимы.
\end{frame}

\setlength{\parskip}{0.5em}

\begin{frame}
\frametitle{Модель с непрерывным временем} 
В глобальном методе определения пуассоновского процесса исходят из того, что $\textbf{P}[N(t+h)-N(t)=k \mid N(s)$ для всех $s \leq t] = \frac{e^{- \lambda h}(\lambda h)^{k}}{k!}$ для всех $t\geq 0$ и $h>0 ,$  $k = 0, 1, 2, ...$
Из этого определения вытекают следующие свойства:

1) Приращения стационарны.

2) Для любого множества непересекающихся временных интервалов приращения независимы, то есть для $t_{1}<t_{1}+h{1}<t_{2}<t_{2}+h{2}<...<t_{n}+h_{n}$ приращения $N(t_{1}+h_{1})-N(t_{1}), N(t_{2}+h_{2})-N(t_{2}),...\:,N(t_{n}+h_{n})-N(t_{n})$ взаимно независимы.

3)Вероятность того, что несколько страховых случаев произойдет одновременно, равна нулю, то есть:

$ \lim_{h \rightarrow 0} \: \frac{\textbf{P}[N(t+h)-N(t)>1]}{h} = \lim_{h \rightarrow 0} \: \frac{1-e^{- \lambda h}- \lambda he^{- \lambda h}}{h} = 0$
\end{frame}

\begin{frame}
\frametitle{Модель с непрерывным временем} 

Перейдем к определению сложного пуассоновского процесса. Если для $S(t)$ случайные величины $X_{1}, X_{2}, X_{3}, ...$ независимы, имеют общую функцию распределения $P(x)$ и если они также независимы от процесса $\big\{N(t), t\geq 0\big\}$, то процесс $\big\{S(t), t\geq 0\big\}$ называется сложным пуассоновским процессом. Обладает несколькими свойствами. Вот одно из них:

Если $t\geq 0 $ и $h > 0$, то распределение c.в. $S(t + h) - S(t)$ является сложным пуассоновским c параметром $\lambda h$ функцией распределения $P(x)$, то есть:
\setlength{\parindent}{0ex}

$\textbf{P}[S(t+h)-S(t)\leq x \mid S(s)$  $\forall s \leq t] = $ 

$= \sum_{k=0}^{ \infty }e^{- \lambda h}(\lambda h)^{k} \frac{P^{\ast k}(x) }{k!}$
\end{frame}

\begin{frame}
\section{Понятие разорения}
\frametitle{Некоторые понятия}
\begin{block}{Определение 1}
Рассмотрим период длины $t > 0$, где размер собранной  премии равен $ct$, а $S(t)$ - величина суммарных страховых выплат, распределение которой - сложное пуассоновское, причём $\mathbb{E}N(t) = {\lambda}t$.
\\ \textbf{Коэффициент Лундберга $\widetilde{R}$} - наименьшее положительное решение уравнения $$M_{S(t)-ct}(r) = \mathbb{E}[e^{r(S(t)-ct)}] = e^{-rct}M_{S(t)}(r) = $$ $$= e^{-rct}e^{-{\lambda}t[M_{X}(r)-1]} = 1 \Leftrightarrow {\lambda}t[M_{X}(r)-1] = cr,$$или, при $c = (1+{\theta}){\lambda}p_{1}r$, уравнения 
\begin{center}$1+(1+\theta){\lambda}p_{1}r = M_{X}r$ \textbf{(2.1)} \end{center}
\end{block}
\begin{block}{Определение 2}
\textbf{Вероятность разорения $\phi(u)$} - 
\end{block}
\end{frame}


\begin{frame}
\section{Вычисление вероятности разорения}
\frametitle{Вычисление вероятности разорения}
Рассмотрим случай с показательным распределением величины страховых выплат с параметром $\beta > 0$.

\textbf{1) Определим коэффициент Лундберга:}

Уравнение 2.1 принимает вид
$$1 + \frac{(1 + \theta)r}{\beta}=\frac{\beta}{\beta-r}$$
или в форме квадратного уравнения по $r$,
$$(1 + \theta)r^2 - \theta\beta r = 0$$
как и ожидалось, $r = 0$ - решение, а наименьшим положительным решением уравнения 2.1
(коэф. Лундберга), оказывается 
$$R = \frac{\theta\beta}{1 + \theta}$$
\end{frame}

\begin{frame}
\frametitle{Вычисление вероятности разорения}
\textbf{2) Вычислим вероятность разорения:}

Пусть разорение, если оно происходит, случается в момент $T$. Пусть $\hat{u}$ является величиной рискового резерва непосредственно перед моментом $T$.

$$\mathbb{P}(-U(T)>y) = \mathbb{P}(X > \hat{u} + y|X > \hat{u}) =
\frac
{\beta\int_{\hat{u}+y}^{\infty} e^{-\beta x}dx}
{\beta\int_{\hat{u}}^{\infty}   e^{-\beta x}dx}
= e^{-\beta y}$$


$$p_{-U(T)|T < \infty}(y) = \frac{d}{dy}(1-e^{-\beta y}) = \beta e^{-\beta y}$$

$$\mathbb{E}[exp(-RU(T))|T<\infty] = \beta\int_{0}^{\infty} e^{-\beta y} e^{Ry}dy =
\frac{\beta}{\beta - R}$$


\end{frame}

\begin{frame}
\frametitle{Вычисление вероятности разорения}

Воспользуемся вычисленным ранее коэффициентом Лундберга и теоремой:
$$\psi(u) = \frac{(\beta-R)e^{-Ru}}{\beta} =
\frac{1}{1+\theta} exp\left(\frac{-\theta\beta u}{1+\theta}\right) =$$
$$=\frac{1}{1+\theta} exp\left[\frac{-\theta u}{(1+\theta)p_1}\right]$$

\end{frame}


\begin{frame}
\frametitle{Использованная литература}
\footnotesize{
\begin{thebibliography}{99}
\bibitem[Н. Бауэрс, Х. Гербер, Д. Джонс, С. Несбитт, Дж. Хикман 1997]{p1} Н. Бауэрс, Х. Гербер, Д. Джонс, С. Несбитт, Дж. Хикман (1997)
\newblock \emph{Актуарная математика}, 355 -- 368.
\end{thebibliography}
}
\end{frame}


\end{document}